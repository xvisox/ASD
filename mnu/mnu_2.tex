\documentclass[a4paper]{article}

\usepackage{amssymb,mathrsfs,amsmath,amscd,amsthm}
\usepackage[mathcal]{euscript}
\usepackage{stmaryrd}
\usepackage[T1]{fontenc}
\usepackage[utf8]{inputenc}
\usepackage[polish]{babel}
\usepackage{graphics}
\usepackage{blindtext}
\usepackage{enumitem}
\usepackage{amsmath}
\usepackage{algorithm}
\usepackage[noend]{algpseudocode}
\usepackage{mathtools}
\usepackage{arydshln}

\makeatletter
\makeatother

\RequirePackage{a4wide}

%%%%%%% makra do notacji

\renewcommand{\le}{\leqslant} %mniejsze bądź równe
\renewcommand{\ge}{\geqslant} %większe bądź równe
\renewcommand\qedsymbol{\scalebox{0.75}{$\blacksquare$}} %koniec dowodu
\newcommand\exendsymbol{\scalebox{1}{$\lrcorner$}} %inny koniec dowodu
\renewcommand{\phi}{\varphi} %litera φ
\newcommand{\eps}{\varepsilon} %litera ε

\newcommand{\N}{\mathbb N} %liczby naturalne
\newcommand{\R}{\mathbb R} %liczby rzeczywiste
\newcommand{\I}{\mathbb I} %liczby rzeczywiste
\newcommand{\set}[1]{\{#1\}} %\set{1,2,3} to zbiór {1,2,3}
\newcommand{\setof}[2]{\{#1\mid #2\}} %\setof{(x,y)}{x,y\in\N,x+y=5} to {(x,y)|x,y∈N, x+y=5}
\newcommand{\from}{\colon} %f\from X\to Y to funkcja f:X→Y

\renewcommand{\subset}{\subseteq} %symbol ⊆
\newcommand{\Longupdownarrow}{\Big\Updownarrow}

\newtheorem{twierdzenie}{Twierdzenie}
\newtheorem{fakt}{Fakt}
\newtheorem{wniosek}{Wniosek}
\newtheorem{lemat}{Lemat}
\newtheorem{zadanie}{Zadanie}
\newtheorem{zadanie*}{Zadanie$^*$}

\title{\vspace{-1cm}
    Metody Numeryczne \\
    \large Praca domowa 1.
}
\author{
    Hubert Michalski hm438596 \\
    \small współpraca: Przemysław Fuchs
}

\begin{document}

\maketitle

\section*{Zadanie 1.2}

Niech $A \in R^{N \times N}$ będzie nieosobliwą macierzą trójdiagonalną. Podaj algorytm, który przy użyciu przekształceń Householdera wyznaczy jej rozkład $QR$ możliwie niskim (jakim?) kosztem. \newline Wskazówka: Być może macierz $Q$ warto wyznaczyć w postaci iloczynu pewnych przekształceń.

\section*{\large Rozwiązanie}

Zapiszmy najpierw standardowy algorytm rozkładu $QR$ za pomocą przekształceń Householdera:

\begin{algorithm}
	\begin{algorithmic}[1]
		\caption{$A=QR$ Householder method (iterative version)}\label{alg:cap2}
		\For{k = 1 : $N$}
		\State Podziel $A=\begin{bmatrix}
		a_{11} & a_{12}\\
		a_{21} & A_{22}
		\end{bmatrix}$, $a_{11} \in \R$

		\State $Q_k =$ m. Householdera t.że $H\cdot \begin{bmatrix}
		a_{11} \\
		a_{21}
		\end{bmatrix} = \begin{bmatrix}
		r_{kk} \\
		0
		\end{bmatrix}$

		\State $\begin{bmatrix}
		r_{12} \\
		B_{22}
		\end{bmatrix} = Q_{k}^{T} \cdot \begin{bmatrix}
		a_{12} \\
		A_{22}
		\end{bmatrix}$

		\State $A = B_{22}$
		\EndFor

		\State \Return $Q = Q_1 \begin{bmatrix}
		1 & \\
		& Q_2
		\end{bmatrix} \cdots \begin{bmatrix}
		I_{k-1} & \\
		& Q_k
		\end{bmatrix} \cdots, R = \begin{bmatrix}
		r_{11} & r_{12} \\
		0 & \ddots
		\end{bmatrix}$
	\end{algorithmic}\label{alg:algorithm2}
\end{algorithm}

Jeśli pokażemy, że kroki 3. i 4. da się wykonać w czasie stałym, to cały algorytm rozkładu $A=QR$ będzie miał złożoność liniową. Jako że na wejściu otrzymujemy liniowo wiele zmiennych, gdzie w oczywisty sposób każda wpływa na wynik, to jest to także ograniczenie dolne i nie da się przedstawić szybszego algorytmu.

Udowodnimy najpierw, że macierz Householdera taką, jak w kroku 3. da się wyznaczyć w $\mathcal{O}(1)$. Pamiętamy, że m. Householdera można reprezentować za pomocą wektora $v$ takiego, że $H=I- \gamma vv^T$ gdzie $\gamma = 2/ ||v||_{2}^{2}$. Zauważmy, że dany wektor: $$\vec{a}=\begin{bmatrix}
a_{11} \\
a_{21}
\end{bmatrix}$$ ma jedynie dwie niezerowe wartości, ponieważ macierz $A$ jest trójdiagonalna. Potem dokładnie pokażemy, że ta własność macierzy jest zachowana w każdej iteracji. Żeby wyznaczyć wektor $v$ możemy skorzystać ze wzoru $v=\vec{a} + sgn(a_{11})||\vec{a}||_2 \vec{e_1}$, gdzie policzenie normy to koszt $\mathcal{O}(2)$ ponieważ tylko pierwsze dwa elementy są niezerowe, co ostatecznie daje nam koszt stały dla tego kroku algorytmu. Dodatkowo zwróćmy uwagę na to, że nie przedstawiamy macierzy $Q_k$ dokładnie (tzn. jako faktycznie macierzy) a jedynie jako wektor który ma dwa elementy lub pamiętamy go jako dwie stałe.

Aby udowodnić, że 4. krok algorytmu da się wykonać w czasie stałym, przyjrzyjmy się najpierw postaci macierzy $Q_k$:

$$Q_k=I-\gamma vv^T=\begin{bmatrix}
1 &  & \dots & 0 \\
& 1 &  & \vdots \\
\vdots &  & \ddots &  \\
0 & \dots &  & 1
\end{bmatrix} - \gamma  \begin{bmatrix}
v_1 \\
v_2 \\
0 \\
\vdots \\
0
\end{bmatrix}   \begin{bmatrix}
v_1 & v_2 & 0 & \dots & 0
\end{bmatrix} =
\begin{bmatrix}
	1 - \gamma v_1^2 & -\gamma v_1v_2   & \dots  & 0      \\
	-\gamma v_1v_2   & 1 - \gamma v_2^2 &        & \vdots \\
	\vdots           &                  & \ddots &        \\
	0                & \dots            & \dots  & 1
\end{bmatrix}$$
Widzimy, że po przemnożeniu z lewej przez $Q_k$ jedynie dwa pierwsze wiersze macierzy zostaną zmienione, co już potencjalnie zmniejsza ilość obliczeń potrzebnych do uzyskania wyniku. Możemy jednak posunąć się jeszcze o krok dalej i przeanalizować jakiej postaci jest cały iloczyn:
$$\begin{bmatrix}
r_{12} \\
B_{22}
\end{bmatrix} = Q_{k}^{T} \cdot \begin{bmatrix}
a_{12} \\ \hdashline[2pt/2pt]
A_{22}
\end{bmatrix} =
\begin{bmatrix}
	1 - \gamma v_1^2 & -\gamma v_1v_2   & \dots  & 0      \\
	-\gamma v_1v_2   & 1 - \gamma v_2^2 &        & \vdots \\
	\vdots           &                  & \ddots &        \\
	0                & \dots            & \dots  & 1
\end{bmatrix}
\begin{bmatrix}
	\tau    & 0       & \dots   & 0      \\ \hdashline[2pt/2pt]
	a'_{11} & a'_{12} &         &        \\
	a'_{21} & a'_{22} & a'_{23} &        \\
	        & a'_{32} & a'_{33} & \ddots \\
	        &         & \ddots  & \ddots
\end{bmatrix}
$$
gdzie $a'_{ij}$ to elementy macierzy $A_{22}$, a $\tau$ to jedyny element wektora $a_{12}$ z przedstawionego uprzednio pseudokodu. Łatwo teraz zauważyć, że obliczenia jakie trzeba wykonać to:
$$
r'_1 = (1 - \gamma v_1^2)\cdot \tau -\gamma v_1v_2 \cdot a'_{11}
$$
$$
r'_2 = -\gamma v_1v_2 \cdot a'_{12}
$$
więc wynikowy wektor to
$$
r_{12} = \begin{bmatrix}
r'_1 & r'_2 & 0 & \dots & 0
\end{bmatrix}
$$
następnie liczymy pierwszy wiersz macierzy $B_{22}$:
$$
b_{11} = (-\gamma v_1v_2) \cdot \tau + (1 - \gamma v_2^2) \cdot a'_{11}
$$
$$
b_{12} = (1 - \gamma v_2^2) \cdot a'_{12}
$$
Zatem jesteśmy w stanie wykonać iloczyn zadany w kroku 4. w czasie stałym, ponieważ nie przepisujemy całej macierzy do $B_{22}$ a jedynie nadpisujemy pierwszy wiersz macierzy $A_{22}$. Warto także zauważyć, ze macierz $B_{22}$ także będzie diagonalna zatem kolejne iteracje będą przebiegały analogicznie. Dodatkowo wynikowy wektor $r_{12}$ posiada jedynie dwie stałe, jest to istotne ponieważ w każdej iteracji do macierzy wynikowej $R$ dokładamy jedynie stałą ilość zmiennych więc będzie ich liniowo wiele względem wejścia.

Podsumowując, jesteśmy w stanie poznać rozkład $A=QR$ dla macierzy trójdiagonalnej nieosobliwej w czasie $\mathcal{O}(N)$ jeśli będziemy reprezentować macierze Householdera $Q_k$ za pomocą ich wektorów $v_k$, gdzie będziemy pamiętać jedynie niezerowe wartości. Analogicznie dla macierzy $R$ - pamiętamy jedynie niezerowe wartości w wierszach a nie całe wektory.

\end{document}
